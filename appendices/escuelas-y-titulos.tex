\chapter{Escuelas y títulos}
\label{ch:escuelas-y-titulos}

A continuación se describen todas las opciones de grados y títulos disponibles en la memoria.

\section{Escuelas}

Las escuelas disponibles se describen en el cuadro~\ref{tbl:schools}.

\begin{table}
    \centering
    \begin{tabularx}{\textwidth}{@{}lX@{}}
        \toprule
        \textbf{Clave}  & \textbf{Valor} \\
        \midrule
        \texttt{etsii} & E.T.S. de Ingeniería Informática \\
        \bottomrule
    \end{tabularx}
    \caption{\label{tbl:schools} Relación entre el código de la plantilla y la escuela a la que se refiere}
\end{table}

De momento no están todas, así que si te apetece añadir la tuya puedes, o bien contactar con los autores, o bien modificarlo (mira el apéndice~\ref{ch:ampliar}) y también contactar con los autores, así lo podemos hacer público con la mayor cantidad de usuarios posible.

\section{Identidad visual corporativa de la URJC }

A la hora de realizar cualquier trabajo o material debemos cuidar ciertos aspectos relacionados con nuestra insticución. En concreto, deberíamos seguir el Manual de Identidad Corporativa de nuestra universidad. Este  manual reúne las herramientas básicas para el correcto uso y aplicación
gráfica de la marca Universidad Rey Juan Carlos en todas sus posibles expresiones.

Uno de los elementos que permiten identificar a nuestra universidad fácilmente, más allá de los logotipos es el color. Concretamente, nuestro rojo es el PANTONE 485 C, (R203 G0 B23 t y HTML \#CB0017).

\subsection{E.T.S. de Ingeniería Informática}

Las titulaciones que existen la última vez que se actualizó este documento son las siguientes:

\begin{itemize}
    \item \texttt{GCID}: Grado en Ciencia e Ingeniería de Datos
    \item \texttt{GDDV}: Grado en Diseño y Desarrollo de Videojuegos
    \item \texttt{GIA}: Grado en Inteligencia Artificial
    \item \texttt{GIC}: Grado en Ingeniería de Computadores
    \item \texttt{GICIB}: Grado en Ingeniería de la Ciberseguridad
    \item \texttt{GII}: Grado en Ingeniería Informática
    \item \texttt{GIS}: Grado en Ingeniería del Software
    \item \texttt{MAT}: Grado en Matemáticas
\end{itemize}

