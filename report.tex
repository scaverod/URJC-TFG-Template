\documentclass[%
    school=etsii,%
    type=tfg,%
    degree=MAT,%
    authorsex=m,%
    directorsex=m,%
]{urjc-report}

\addbibresource{references.bib}

\title{Plantilla de Trabajos Fin de Grado de la ETSII de la URJC}
\author{Autora Proyecto}
\bibauthor{Proyecto, A.}
\director{Directora Proyecto}
\bibdirector{Proyecto, D.}

\codirector[m]{Codirectora Proyecto}{Proyecto, C.}

\githuburl{https://github.com/yourusername/yourrepository}


\abstract{spanish}{
    El resumen de un \acrlong{tfg} o de un \acrlong{tfm} condensa en tres o cuatro párrafos el contenido de la memoria. Se debe dar por sentado que el lector podrá tener una idea clara de lo que trata, y suele ser la primera barrera donde decide si continúa leyendo o no el texto.
    
    \textbf{Condensado no quiere decir incompleto}. Debe contener la información más destacable. Lo ideal es que ocupe entre media y una cara de un folio A4. Comenzará por el propósito y principales objetivos de la memoria. Luego hablaremos sobre los aspectos más destacables de la metodología empleada, seguido de los resultados obtenidos. Por último se presentarán las conclusiones de forma condensada.
    
    Debe tener un estilo claro y conciso, sin ambigüedades de ningún tipo. Además, al ser un resumen de todo el contenido, ni que decir tiene que deberá ser lo último que elaboraremos, y deberá mantener una absoluta fidelidad con el contenido de la memoria.
}
\keywords{spanish}{Cuatro o cinco; Expresiones clave; Que resuman; Nuestro proyecto o; Investigación}

\abstract{english}{
    This section must contain the summary that we have written before in Spanish, but in English, as well as the keywords.
}
\keywords{english}{Four or five; Key Expressions; Summarising; Our Project or; Research}

\acknowledgements{
    Aquí los agradecimientos que quieras dar. Y si no quieres, borras la entrada \texttt{\textbackslash acknowledgements} de \texttt{report.tex} y ya está.
}

\begin{document}

\include{frontmatter/glossary}

\input{chapters/introduccion}
\input{chapters/configuracion}
\chapter{Componentes de la plantilla}
\label{ch:componentes-de-la-plantilla}

En este capítulo hablaremos de los componentes principales con los que trabajaremos en nuestra memoria.

\section{Columnas}

TBD

\section{Ecuaciones}

La facilidad de composición de ecuaciones es una de las cosas que más atrae de \LaTeX\space a muchos autores. \LaTeX mantiene dos renderizadores diferentes, uno para el texto y otro para las ecuaciones, denominados modo párrafo y modo matemático\footnote{Existe un tercer modo, denominado \textit{LR mode} o \textit{left-to-right mode}, raramente utilizado y que no trataremos aquí}. El modo párrafo es el modo por defecto y no se le llama explícitamente. Al modo matemático, sin embargo, se le invoca de varias maneras diferentes.

\subsection{Modo en párrafo}

La forma más común es la forma ``en línea'', donde el texto para el modo matemático se encierra entre dos signos \$. Por ejemplo, veamos la frase del listado~\ref{lst:in-line-equation}.

\begin{lstlisting}[language=tex,caption=Ejemplo de inserción de fórmulas en linea,label=lst:in-line-equation]
El pequeño teorema de Fermat dice que si $p$ es un número primo, entonces, para cada número natural $a$, con $a>0$, $a^p \equiv a (\mod p)$
\end{lstlisting}

La frase quedaría como sigue:

El pequeño teorema de Fermat dice que si $p$ es un número primo, entonces, para cada número natural $a$, con $a>0$, $a^p \equiv a (mod p)$

\subsection{Ecuaciones en bloque}

Cuando en lugar de poner una ecuación dentro de un párrafo existente la queremos insertar en su propio espacio independiente hacemos uso de los entorno \texttt{equation} o \texttt{align}, dependiendo de si queremos una o más ecuaciones en el bloque, respectivamente. Por ejemplo, en el caso de una única ecuación, sería similar al ejemplo siguiente:

\begin{minipage}[c]{.5\textwidth}
\begin{lstlisting}[language=tex]
\begin{equation}
	\int_a^b f'(x) \, dx=f(b)-f(a)
\end{equation}
\end{lstlisting}
\end{minipage}%
\begin{minipage}[c]{.5\textwidth}
\begin{equation}
    \int_a^b f'(x) \, dx=f(b)-f(a)
\end{equation}
\end{minipage}

Sin embargo, en el caso de que quisiésemos más de una ecuación en el mismo bloque haríamos uso del carácter \& para indicar en qué punto se alinean las ecuaciones; por ejemplo:

\begin{minipage}[t]{.5\textwidth}
\begin{lstlisting}[language=tex]
\begin{align}
    u  &= \arctan x \\ 
    du &= \frac{1}{1 + x^2}dx
\end{align}
\end{lstlisting}
\end{minipage}%
\begin{minipage}{.5\textwidth}
\begin{align}
    u  &= \arctan x \\ 
    du &= \frac{1}{1 + x^2}dx
\end{align}
\end{minipage}

Por último, si no tenemos por qué referenciarlas en el texto, podemos hacer uso de los entornos \texttt{equation*} y \texttt{align*}

\begin{minipage}[c]{.5\textwidth}
\begin{lstlisting}[language=tex]
\begin{equation*}
	\int_a^b f'(x) \, dx=f(b)-f(a)
\end{equation*}
\end{lstlisting}
\end{minipage}%
\begin{minipage}[c]{.5\textwidth}
\begin{equation*}
	\int_a^b f'(x) \, dx=f(b)-f(a)
\end{equation*}
\end{minipage}

\section{Elementos flotantes}

Vamos a hablar un poco de los elementos denominados \enquote{flotantes} o \textit{floats}. Éstos son bloques de contenido que \enquote{flotan} por la página hasta que \hologo{LaTeX} los coloca donde considera a través de ciertos algoritmos.

Lo general es \textbf{declarar el elemento flotante inmediatamente después del párrafo donde se ha referenciado} (las referencias cruzadas las vemos en la~\autoref{s:crossref} de este capítulo), y después dejar que \LaTeX\space elija el mejor sitio. Y si después de haber escrito todo el documento hay algo que no cuadre, pues ahí modificarlo.

\enquote{¿Y si no hago referencia a un elemento flotante, dónde lo pongo?} Bueno, pues en general no se pone. Si algo no es nombrado, suele ser porque no aporta información, y si no la aporta, pues para qué lo vamos a meter. Ojo, que lo mismo existe algún caso donde sí tiene sentido, pero es muy raro.

Aunque más adelante veremos los diferentes tipos de \textit{floats}, en caso de que queramos modificar su comportamiento todos tienen los especificadores de posición indicados en la tabla~\ref{tab:floats-options}:

\begin{table}
    \caption{\label{tab:floats-options}Opciones para los elementos flotantes de \hologo{LaTeX}}
    \centering
    \begin{tabularx}{\textwidth}{@{}lX@{}}
        \toprule
        \textbf{Especificador} & \textbf{Acción} \\
        \midrule
        H & Colocar exactamente en el sitio indicado                       \\
        h & Colocar aproximadamente en el sitio indicado                   \\
        t & Colocar al comienzo de la página                               \\
        b & Colocar al final de la página                                  \\
        p & Colocar en una página exclusiva para elementos ``flotantes''   \\
        ! & Forzar las opciones obviando los mecanismos internos de \LaTeX \\
    \bottomrule
    \end{tabularx}
\end{table}

\subsection{Código fuente}

Para la gestión de los listados de código fuente se utiliza el paquete \texttt{listings}. El estilo usado es una modificación\footnote{En realidad la modificación es cambiar el fondo de crema a blanco} de \href{https://ethanschoonover.com/solarized/}{Solarized} desarrollado por Ethan Schoonover.

Existen muchas formas diferentes de incluir listados de código en una memoria. Aquí introducimos los más comunes.

\subsubsection{Código dentro del propio párrafo}

TBD

\subsubsection{Bloques de código}

Insertar código en párrafos no es tan común como insertar bloques enteros. Para ello haremos uso del entorno \texttt{lstlisting}. Por ejemplo:

\lstinputlisting[language=tex, firstline=1, lastline=6]{sources/adding-blocks.tex}

Nos daría el siguiente resultado:

\lstinputlisting[language=python, firstline=2, lastline=5]{sources/adding-blocks.tex}

Una cosa que hay que tener en cuenta es que dentro de un entorno \texttt{lstlisting} se ignoran todos los comandos de \LaTeX\space\footnote{En realidad no todos, y si no mira en estos fuentes cómo hemos metido el fin de bloque \texttt{lstlisting} dentro del propio bloque.} y el texto se imprime tal y como se ha introducido. Esto incluye los tabuladores y espacios de principio de línea.

Al igual que en el código de párrafo, también podemos especificar en qué lenguaje está escrito el código para que se resalten en éste las palabras reservadas. Por ejemplo:

\lstinputlisting[language=tex, firstline=8, lastline=14]{sources/adding-blocks.tex}

Nos daría como resultado el siguiente bloque de código:
 
\lstinputlisting[language=python, firstline=9, lastline=12]{sources/adding-blocks.tex}

\subsubsection{Código directamente desde fichero}

Esta forma es muy común, ya que se usa tanto para hacer referencia al código fuente de la aplicación directamente, como a código separado en ficheros para mantener el tamaño de la memoria manejable.

Suponiendo que tenemos el fichero \texttt{sources/snippets.py}, para incluirlo entero basta con usar el comando \texttt{lstinputlisting}:

\begin{lstlisting}[language=TeX]
\lstinputlisting[language=python]{sources/snippets.py}
\end{lstlisting}

Con este comando conseguiríamos el siguiente resultado:

\lstinputlisting[language=python]{sources/snippets.py}

En caso de que se desease importar sólo parte del fichero, se pueden indicar las filas que delimitan el trozo de código. Por ejemplo:

\begin{lstlisting}[language=TeX]
\lstinputlisting[
    language=python,
    firstline=4,
    lastline=5
]{sources/snippets.py}
\end{lstlisting}

Esto haría que sólo se imprimiesen las filas 4 y 5, correspondientes a la segunda función:

\lstinputlisting[language=python, firstline=4, lastline=5]{sources/snippets.py}

Ambas opciones son opcionales, y los valores por defecto de \texttt{firstline} y \texttt{lastline} serán el principio y el final del fichero respctivamente.

\subsubsection{Etiquetando bloques de código}

Al igual que con el resto de bloques \textit{float} (e.g. figuras o tablas), se pueden\footnote{Pongo \textit{pueden} porque es opcional, pero en realidad se \textbf{deben} poner, porque si no los listados de fuentes quedan horribles, como el de esta plantilla por ejemplo (échale un vistazo si no lo has hecho antes).} (\textbf{deben}) añadir pies de texto a los bloques de código, lo cual hace más legible su cometido.

Para ello basta con añadir el argumento \texttt{caption} a las opciones del bloque. Por ejemplo:

\lstinputlisting[language=tex, firstline=15, lastline=20]{sources/adding-blocks.tex}

Nos daría como resultado el siguiente bloque de código:
 
\lstinputlisting[language=python, firstline=16, lastline=19, caption=Función para determinar cuando una palabra \texttt{w1} es anagrama de otra palabra \texttt{w2}]{sources/adding-blocks.tex}

\subsection{Figuras}

El código siguiente (listado~\ref{lst:figure-basic-insert}) renderizará una imagen.

\begin{lstlisting}[language=python, caption=Inserción de una figura,label={lst:figure-basic-insert}]
\begin{figure}[H]
    \centering
    \includegraphics[width=0.25\textwidth]{figures/vault-boy.png}
    \caption{\label{fig:img-vault-boy} Vault Boy approves that}
\end{figure}
\end{lstlisting}

La sintaxis es bastante autoexplicativa. El entorno \texttt{figure} es el que delimita el contenido de la figura. El comando \texttt{centering} determina que se tiene que centrar . Luego, el comando \texttt{caption} determina el pie de imagen, el cual además incluye una etiqueta (comando \texttt{label}) que sirve para referenciar. Por último, se incluye la imagen con el comando \texttt{includegraphics}.

En definitiva, la imagen (figura~\ref{fig:img-vault-boy}) se mostrará con un ancho igual al 25\% del ancho que ocupa el texto, centrada, con un pie de foto y una etiqueta para referenciar.

\begin{figure}
    \centering
    \includegraphics[width=0.25\textwidth]{figures/vault-boy.png}
    \caption{\label{fig:img-vault-boy} Vault Boy approves that}
\end{figure}

El comando \texttt{includegraphics} puede importar los formatos típicos de imagen, como jpeg, png o pdf. También admite una serie de opciones como rotación, alto, ancho (éste le hemos especificado con \texttt{width}), etcétera. ¡Ojo! Siempre que se pueda, hay que intentar insertar imágenes vectoriales. De esta manera, se mantiene la calidad de la imagen. Si no, puede ocurrir que se pixele y no quede nada bien.

\subsubsection{Subfiguras}

¿Y qué pasa cuando queremos incluir múltiples imágenes dentro de una figura? Bueno, pues aquí hay que usar el entorno \texttt{subfigure}. En el listado~\ref{lst:subfigure-basic-insert} vemos un ejemplo de cómo se manejan.

\begin{lstlisting}[language=python, caption=Inserción de varias subfiguras,label={lst:subfigure-basic-insert}]
\begin{figure}[H]
	\centering
	\begin{subfigure}{.3\textwidth}
		\includegraphics[width=\linewidth]{figures/vault-boy.png}
		\caption{\label{fig:subfigure-1}Vault Boy 1}
	\end{subfigure}%
	\begin{subfigure}{.3\textwidth}
		\includegraphics[width=\textwidth]{figures/vault-boy.png}
		\caption{Vault Boy 2}
	\end{subfigure}%
	\begin{subfigure}{.3\textwidth}
		\includegraphics[width=\textwidth]{figures/vault-boy.png}
		\caption{Vault Boy 3}
	\end{subfigure}
	\caption{\label{fig:subfigures}Todos los Vault Boy}
\end{figure}
\end{lstlisting}

En realidad cada subfigura se trata como una figura normal, pero en relación con el \textit{float} contenedor. Cuando a una subfigura se le especifica un ancho, se le está diciendo al compilador de qué ancho es esa subfigura en concreto (en nuestro caso 0.3 veces el ancho de la línea). Sin embargo, a la imagen se le da un ancho total de \texttt{linewidth}, porque al estar dentro de su espacio de subfigura, el ancho ha cambiado. El resultado es el que se observa en la figura~\ref{fig:subfigures}. Por cierto, también podemos referenciar a los pies de las subfiguras (e.g. Así:~\ref{fig:subfigure-1}).

\begin{figure}
	\centering
	\begin{subfigure}{.3\textwidth}
		\includegraphics[width=\linewidth]{figures/vault-boy.png}
		\caption{\label{fig:subfigure-1}Vault Boy 1}
	\end{subfigure}%
	\begin{subfigure}{.3\textwidth}
		\includegraphics[width=\textwidth]{figures/vault-boy.png}
		\caption{Vault Boy 2}
	\end{subfigure}%
	\begin{subfigure}{.3\textwidth}
		\includegraphics[width=\textwidth]{figures/vault-boy.png}
		\caption{Vault Boy 3}
	\end{subfigure}
	\caption{\label{fig:subfigures}Todos los Vault Boy}
\end{figure}

\subsection{Tablas}

Las tablas (en realidad \enquote{cuadros}) son una forma muy eficaz de presentar información. En los resultados de casi cualquier trabajo existen cuadros de algún tipo para que los datos se comprendan de un único vistazo (o para que al menos sea más fácil identificarlos.

\notebox{
    \textbf{¿Por qué ``cuadro'' en lugar de ``tabla''?}
    
    Tal y como se indica en el \href{http://www.aq.upm.es/Departamentos/Fisica/agmartin/webpublico/latex/FAQ-CervanTeX/FAQ-CervanTeX-6.html}{FAQ de CervanTex}, \textit{table} (inglés) y \textit{tabla} (español) son falsos amigos; el inglés \textit{table} tiene un sentido más general que el español \textit{tabla}, cuyo uso es únicamente para aquellos cuadros dedicados a la disposición de números (e.g. tabla de multiplicar o tabla de logaritmos).
}

Sin embargo, las tablas suelen ser bastante complicadas en \LaTeX, por lo menos para la gente que empieza. Para no escribir demasiado, la respuesta para casi toda maquetación de tabla está en \href{https://www.tablesgenerator.com/}{https://www.tablesgenerator.com/}. En serio, no perdáis el tiempo si no es estrictamente necesario. La maquetáis visualmente, la generáis (con estilo \textit{booktabs}) y a correr.

\subsection{Pseudocódigos}


\begin{algorithm}
\begin{algorithmic}[1]
\STATE $\forall i \in V$, \ let $i$ be an isolated community
\STATE $o=\texttt{permutation}(V)$
\FOR{$k \ \in \ o$}
\STATE search in $A$ all the neighbours of $k$, $j$
\STATE $\forall j$, calculate $\Delta Q_k(j)$ in matrix $\mathcal{M}$
\STATE $j^*=\{ \ j \ | \ \Delta Q_k(j^*)=\max_j\{Q_k(j)\} \ \}$
\IF{$\Delta Q_k(j^*)>0$}
\STATE{Move node $k$ to $j^*$ 's community}
\ELSE
\STATE{$k$ remains in its community}
\ENDIF
\ENDFOR
\end{algorithmic}\caption{\textit{Additional Louvain} \textbf{input}=$\left(A, \ \mathcal{M}\right)$ \textbf{output}=$P$}
\label{alg:AdditionalLouvain}
\end{algorithm}

En el algoritmo~\ref{alg:AdditionalLouvain} aparece un ejemplo en pseudocódigo.

\section{Enlaces de hipertexto}

TBD

\section{Fórmulas matemáticas}

TBD

\section{Glosario}
\label{s:glosario}

El glosario de una memoria es el lugar donde se encuentran los términos que se usan a lo largo del documento y que se considera que requieren una aclaración. En esta plantilla, en el momento que generemos un término, se creará un capítulo al final de la memoria con el listado de todos aquellos términos definidos.

Para gestionar el glosario se hace uso del paquete \texttt{glossaries} el cual es relativamente complejo de configurar. También su documentación es muy extensa\footnote{Pero aún así es el sitio donde ir a buscar información. El paquete, junto con su documentación está disponible en la dirección \href{https://www.ctan.org/pkg/glossaries}{https://www.ctan.org/pkg/glossaries}.}, así que en esta sección hablaremos únicamente de lo esencial.

\subsection{¿Cuándo y cómo especificar términos?}

La regla general del \enquote{cuándo} es una vez terminada la memoria. En ese punto, seremos conscientes de qué términos son los más interesantes para incluir en el glosario. En ese punto deberemos ir término por termino sustituyéndolo por la entrada del glosario para que el proceso automático se encargue de la indexación y numeración de páginas.

El \enquote{cómo} se refiere a de qué manera escribirlos. La regla general en el castellano (y hasta donde el autor de la plantilla sabe, en cualquier idioma) es de la manera en la que aparecería en medio del texto. Es decir, si la palabra se escribe generalmente en minúscula (e.g.~\textit{El jugador blandía un \gls{hacha-batalla}}) se deberá incluir dentro del glosario en minúscula, mientras que si se escribe generalmente en mayúscula (e.g.~\textit{Encontró el \gls{arco-perdicion}}) irá en mayúscula.

\subsection{Definiendo los términos del glosario}

Las entradas se escribirán dentro del fichero \texttt{frontmatter/glossary.tex}. La forma estándar de definir un término es la que se muestra en el listado~\ref{lst:std-glossary-entry}.

\begin{lstlisting}[language={[latex]TeX},caption=Código para crear una entrada en el glosario,label=lst:std-glossary-entry]
\newglossaryentry{hacha-batalla}{
    name={hacha de batalla},
    description={Herramienta antigua utilizada en combate}
}
\end{lstlisting}

Luego, dentro del texto, podremos hacer referencia a dichas entradas con los comandos que se muestran en la tabla~\ref{tab:glossary-commands}.

\begin{table}
    \caption{\label{tab:glossary-commands}Comandos para incluir términos del glosario en el texto de la memoria}
    \begin{tabularx}{\textwidth}{@{}lX@{}}
        \toprule
        \textbf{Comando} & \textbf{Ejemplo con la clave \texttt{hacha-batalla}} \\
        \midrule
        \texttt{\textbackslash gls} & \gls{hacha-batalla} \\
        \texttt{\textbackslash Gls} & \Gls{hacha-batalla} \\
        \texttt{\textbackslash glspl} & \glspl{hacha-batalla} \\
        \texttt{\textbackslash Glslp} & \Glspl{hacha-batalla} \\
        \bottomrule
    \end{tabularx}
\end{table}

Como los plurales los gestiona automáticamente, puede ser que queramos, como en este caso, modificar el plural de nuestro término. Para ello debemos añadir la opción \texttt{plural} a la entrada para especificar cómo es el plural de la entrada, como se muestra en el listado~\ref{lst:gls-longplural}.

\begin{lstlisting}[language={[latex]TeX},caption=Especificando el plural para un término del glosario,label=lst:gls-longplural]
\newglossaryentry{python}{
    name={Python},
    plural={Pythonacos},
    description={El mejor lenguaje de programación}
}
\end{lstlisting}

Así, el plural de la clave \texttt{python} descrita quedaría como \glspl{python}, en lugar del valor por defecto que sería \textit{Pythons}.

Un caso particular de términos del glosario son las siglas y los acrónimos. No vamos a entrar en detalle aquí\footnote{Pero recomendamos visitar \href{https://www.fundeu.es/recomendacion/siglas-y-acronimos-claves-de-redaccion/}{https://www.fundeu.es/recomendacion/siglas-y-acronimos-claves-de-redaccion/} y darle una leída porque es interesante.} sino que vamos a introducir las siglas como caso especial de entrada de glosario. Cuando tengamos una sigla, la crearemos en el glosario como se muestra en el listado~\ref{lst:new-acronym}.

\begin{lstlisting}[language={[latex]TeX},caption=Entrada genérica de una sigla o acrónimo en el glosario,label=lst:new-acronym]
\newacronym[
    description={Proyecto Fin de Grado. Proyecto a realizar al final de una titulación de Grado},
    longplural={Proyectos Fin de Grado}
    ]{tfg}{tfg}{Proyecto Fin de Grado}
\end{lstlisting}

En el ejemplo se puede ver que hay dos entradas, \texttt{longplural} y \texttt{description} que son opcionales. La primera es la equivalente a \texttt{plural} de \texttt{newglossaryentry}, y no necesita más explicación.

La segunda, \texttt{description} suele utilizarse para acrónimos, cuando necesitamos describir la entrada. Cuidado en este caso porque si hace referencia a varias palabras estas se deberían incluir dentro de la descripción (como en el ejemplo, \enquote{\acrlong{tfg}}).

La regla general de los acrónimos y las siglas es que la primera vez que aparecen en el texto, deben aparecer con el nombre completo mientras que el resto de veces pueden aparecer indistintamente como sigla o forma larga. De esto se encarga automáticamente el comando \texttt{gls}. Es decir, si tenemos la sigla \texttt{special}, la primera vez que incluyamos la sigla con \texttt{\textbackslash gls\{special\}} saldrá \gls{special} mientras que el resto de veces que la incluyamos se verá simplemente \gls{special}.

Con los acrónimos se incluyen comandos adicionales para controlar su presentación. Estos son los mostrados en la tabla~\ref{tab:acronym-commands}

\begin{table}
    \caption{\label{tab:acronym-commands}Comandos específicos para controlar la presentación de acrónimos}
    \begin{tabularx}{\textwidth}{@{}lX@{}}
        \toprule
        \textbf{Comando} & \textbf{Ejemplo con la clave \texttt{rpg}} \\
        \midrule
        \texttt{\textbackslash acrshort} & \acrshort{rpg} \\
        \texttt{\textbackslash acrshortpl} & \acrshortpl{rpg} \\
        \texttt{\textbackslash acrlong} & \acrlong{rpg} \\
        \texttt{\textbackslash Acrlong} & \Acrlong{rpg} \\
        \texttt{\textbackslash acrlongpl} & \acrlongpl{rpg} \\
        \texttt{\textbackslash Acrlongpl} & \Acrlongpl{rpg} \\
        \texttt{\textbackslash acrfull} & \acrfull{rpg} \\
        \texttt{\textbackslash Acrfull} & \Acrfull{rpg} \\
        \texttt{\textbackslash acrfullpl} & \acrfullpl{rpg} \\
        \texttt{\textbackslash Acrfullpl} & \Acrfullpl{rpg} \\
        \bottomrule
    \end{tabularx}
\end{table}

Por cierto, en castellano las siglas \textbf{no incluyen la \enquote{s} al final}, así que no deberíamos usar los comandos que terminan en \texttt{pl}. Por eso la definición que se ha hecho de la sigla \texttt{rpg} es la mostrada en la figura~\ref{fig:acronym-rpg}.

\begin{lstlisting}[language={[latex]TeX},caption=Entrada de \texttt{rpg} en \texttt{glossaries.tex},label=fig:acronym-rpg]
\newacronym[
    description={Role-Playing Game. Juego de rol},
    shortplural={RPG}
    ]{rpg}{RPG}{\textit{Role-Playing Game}}
\end{lstlisting}

\section{Notas}

TBD

\section{Referencias cruzadas}
\label{s:crossref}

Las etiquetas (\textit{label}) son una herramienta muy útil en el proceso de composición tipográfica. Se puede pensar en ellas como punteros a zonas de interés del documento, de tal manera que se les pueda referenciar sin necesidad de conocer su posición final en la composición.

Por ejemplo, lo normal es que en un libro, a la hora de referenciar una figura, aparezca una frase del estilo ``[\ldots] como muestra la Figura 3 [\ldots]''. Lo que es bastante raro son las frases del estilo ``[\ldots] como muestra la Figura de los moñecos amarillos [\ldots]'' o ``[\ldots] como muestra la siguiente Figura [\ldots]''\footnote{Sí, bueno, quizá la segunda frase no es tan rara, pero siempre es preferible referenciar directamente a dar posiciones relativas.}.

Una de las propiedades más útiles y, en ocasiones, infravaloradas de \LaTeX es la facilidad y potencia de su sistema de etiquetado. Este sistema permite referenciar tablas, listados de código fuente, ecuaciones, capítulos, secciones, etc., con facilidad y flexibilidad. Además, \LaTeX las numera y referencia automáticamente, cambiando la numeración en función de las adiciones y supresiones sin que el autor tenga que hacer nada.

Para referenciar un elemento, lo primero que hay que crear es una etiqueta \textbf{después} del elemento a referenciar. Esto ya lo hemos visto anteriormente, por ejemplo en el listado~\ref{lst:figure-basic-insert}. Si nos fijamos, se declara una etiqueta justo después de la etiqueta \texttt{caption} con el nombre \texttt{fig:img-vault-boy}. De esta manera, podemos referenciar varios indicadores de la figura, como se muestra en el listado~\ref{lst:basic-references}

\begin{lstlisting}[language=tex, caption=Referenciando una figura y su página,label={lst:basic-references},]
Mira la Figura~\ref{fig:img-vault-boy} en la página~\nameref{fig:img-vault-boy}.
\end{lstlisting}

Dicho listado daría el siguiente resultado:

\blockquote{Mira la Figura~\ref{fig:img-vault-boy} titulada \nameref{fig:img-vault-boy} en la página~\pageref{fig:img-vault-boy}.}

\notebox{
    \textbf{¿Por qué a mí me aparece el símbolo \texttt{??} en lugar de una referencia?} Pues lo más seguro es que sea un error a la hora de escribir la etiqueta. Menos común, pero también puede pasar, es que el documento no se haya compilado bien. Hay que tener en cuenta que \LaTeX\space fue creado en una época donde las máquinas tenían poca (¡poquísima!) RAM, y para funcionar lo que se hacían eran varias compilaciones sobre el documento, almacenando los valores temporales en ficheros. Y como nadie quiere perder tiempo en cambiar y \textit{debuggear} algo que funciona estupendamente bien, no se reimplementa. De todas formas, si te animas, ahí tienes un buen proyecto que si lo sacas adelante te va a hacer muy famoso.
}

\section{Referencias bibliográficas}
\label{s:referencias-bibliograficas}

Hay muchas formas diferentes de gestionar las referencias bibliográficas, así que aquí hemos decidido elegir una de ellas por considerarla la más cómoda y simple, que es mediante el paquete \textit{biblatex}.

El fichero de referencias, \texttt{references.bib}, incluirá una entrada por cada una de las referencias que se citan durante la memoria. Luego, en el cuerpo del texto, se podrán hacer referencias a dichas entradas y será \LaTeX~después quien se encargue de indexar correctamente, crear los hipervínculos y maquetar automáticamente.

El fichero \texttt{references.bib} puede tener muchas más de las referencias que se citan en el cuerpo del texto. Sin embargo, sólo aparecerán las referencias que se citen en el texto.

\notebox{\textbf{No has dicho en ningún momento \textit{bibliografía}} Sí. Las referencias bibliográficas, también conocidas como lista de referencias o simplemente referencias, son todas aquellas fuentes bibliográficas que han sido citadas a lo largo del documento. La bibliografía, también conocida como referencias externas, es simplemente una lista de recursos utilizados, citados o no. Como generalmente los no referenciados no se usan para dar soporte a un texto científico se suelen descartar.}
 
\subsection{¿Cómo creamos nuevas referencias?}

El fichero \texttt{references.bib} contará cero o más entradas con la estructura mostrada en el listado~\ref{lst:base-bibref}.

\begin{lstlisting}[language=tex,caption=Estructura general de una referencia,label=lst:base-bibref]
@tipo{id,
    author = "Autor",
    title = "Título de la referencia (libro, artículo, enlace, ...)",
    campo1 = "valor",
    campo2 = "valor",
    \ldots
}
\end{lstlisting}

En esta entrada, \texttt{@tipo} indica el tipo de elemento (p. ej. \texttt{@article} para artículos o \texttt{@book} para libros) e \texttt{id} es un identificador \textbf{único en todo el documento} para el elemento. El resto de campos dependerán del tipo de la referencia, aunque generalmente casi todos los tipos comparten los campos de \texttt{author}, \texttt{title} o \texttt{year}.

\subsection{¿De qué manera puedo citar las referencias?}

Para citar las referencias en el texto, puedes utilizar el comando \texttt{\textbackslash cite\{id\}} donde \texttt{id} corresponde al identificador único de la referencia que has definido en el fichero \texttt{references.bib}. Por ejemplo, si deseas citar un trabajo específico, puedes hacerlo de la siguiente manera:

\begin{quote}
Esto es una cita ~\cite{mcculloch1943logical}
\end{quote}

Si necesitas citar múltiples referencias al mismo tiempo, puedes separarlas con comas dentro del comando \texttt{\textbackslash cite\{...\}}. Por ejemplo:

\begin{quote}
O bien, dos citas~\cite{mcculloch1943logical,kotonya1998requirements}.
\end{quote}

Esto generará en el texto las citas correspondientes, y al final del documento, el paquete \textit{biblatex} se encargará de crear la lista de referencias con las entradas que has citado. Recuerda que solo las referencias citadas aparecerán en la lista final, lo que permite mantener el documento ordenado y relevante.
\input{chapters/licencia}

\appendix

\chapter{Escuelas y títulos}
\label{ch:escuelas-y-titulos}

A continuación se describen todas las opciones de grados y títulos disponibles en la memoria.

\section{Escuelas}

Las escuelas disponibles se describen en el cuadro~\ref{tbl:schools}.

\begin{table}
    \centering
    \begin{tabularx}{\textwidth}{@{}lX@{}}
        \toprule
        \textbf{Clave}  & \textbf{Valor} \\
        \midrule
        \texttt{etsii} & E.T.S. de Ingeniería Informática \\
        \bottomrule
    \end{tabularx}
    \caption{\label{tbl:schools} Relación entre el código de la plantilla y la escuela a la que se refiere}
\end{table}

De momento no están todas, así que si te apetece añadir la tuya puedes, o bien contactar con los autores, o bien modificarlo (mira el apéndice~\ref{ch:ampliar}) y también contactar con los autores, así lo podemos hacer público con la mayor cantidad de usuarios posible.

\section{Identidad visual corporativa de la URJC }

A la hora de realizar cualquier trabajo o material debemos cuidar ciertos aspectos relacionados con nuestra insticución. En concreto, deberíamos seguir el Manual de Identidad Corporativa de nuestra universidad. Este  manual reúne las herramientas básicas para el correcto uso y aplicación
gráfica de la marca Universidad Rey Juan Carlos en todas sus posibles expresiones.

Uno de los elementos que permiten identificar a nuestra universidad fácilmente, más allá de los logotipos es el color. Concretamente, nuestro rojo es el PANTONE 485 C, (R203 G0 B23 t y HTML \#CB0017).

\subsection{E.T.S. de Ingeniería Informática}

Las titulaciones que existen la última vez que se actualizó este documento son las siguientes:

\begin{itemize}
    \item \texttt{GCID}: Grado en Ciencia e Ingeniería de Datos
    \item \texttt{GDDV}: Grado en Diseño y Desarrollo de Videojuegos
    \item \texttt{GIA}: Grado en Inteligencia Artificial
    \item \texttt{GIC}: Grado en Ingeniería de Computadores
    \item \texttt{GICIB}: Grado en Ingeniería de la Ciberseguridad
    \item \texttt{GII}: Grado en Ingeniería Informática
    \item \texttt{GIS}: Grado en Ingeniería del Software
    \item \texttt{MAT}: Grado en Matemáticas
\end{itemize}


\input{appendices/ampliar}
\input{appendices/paquetes}
\chapter{Lista de Historias de Usuario}
\label{ch:historias_de_usuario}



\begin{table}[h]
    \centering
    \arrayrulecolor{black}
    \setlength{\arrayrulewidth}{0.8mm}
    \begin{tabular}{|>{\centering\arraybackslash}m{4cm}|m{9cm}|c|}
        \hline
        \multicolumn{2}{|>{\centering\arraybackslash}m{9cm}|}{{\cellcolor{schoolcolor}\color{white}\textbf{<<Nombre de la Historia de usuario>>}} }  & \color{schoolcolor}\textbf{HUXX} \\
        \hline
        \textbf{Usuario} & \multicolumn{2}{m{9cm}|}{<<XXX>>}   \\
        \hline
        \textbf{Sprint} & \multicolumn{2}{m{9cm}|}{<<XXX>>}  \\
        \hline
        \textbf{Prioridad} & \multicolumn{2}{m{9cm}|}{<<Muy Baja/Baja/Media/Alta/Muy Alta>>}  \\
        \hline
        \textbf{Estimación} & \multicolumn{2}{m{9cm}|}{<<XX horas/días/puntos>>}  \\
        \hline
        \textbf{Tiempo dedicado} & \multicolumn{2}{m{9cm}|}{<<XX horas/días/puntos>>}  \\
        \hline
        \textbf{Fecha finalización} & \multicolumn{2}{m{9cm}|}{<<dd/mm/aaaa>>}  \\
        \hline
         \textbf{Descripción} & \multicolumn{2}{m{9cm}|}{\textit{Como} <<usuario>>, \textit{quiero} <<acción>>,  \textit{para} <<beneficio>>.}  \\
        \hline
        \textbf{Criterios de aceptación} & \multicolumn{2}{m{9cm}|}{\begin{itemize}
        \item <<criterio 1>>
        \item <<criterio 2>>
        \end{itemize}}   \\
        \hline
    \end{tabular}
\end{table}


\begin{table}
    \centering
    \arrayrulecolor{black}
    \setlength{\arrayrulewidth}{0.8mm}
    \begin{tabular}{|>{\centering\arraybackslash}m{4cm}|m{9cm}|c|}
        \hline
        \multicolumn{2}{|>{\centering\arraybackslash}m{9cm}|}{{\cellcolor{schoolcolor}\color{white}\textbf{Iniciar sesión en el aula virtual}} }  & \color{schoolcolor}\textbf{HU01} \\
        \hline
        \textbf{Usuario} & \multicolumn{2}{m{9cm}|}{Estudiante de la URJC}   \\
        \hline
        \textbf{Sprint} & \multicolumn{2}{m{9cm}|}{Sprint 1}  \\
        \hline
        \textbf{Prioridad} & \multicolumn{2}{m{9cm}|}{Alta}  \\
        \hline
        \textbf{Estimación} & \multicolumn{2}{m{9cm}|}{3 días}  \\
        \hline
        \textbf{Tiempo dedicado} & \multicolumn{2}{m{9cm}|}{2 días}  \\
        \hline
        \textbf{Fecha finalización} & \multicolumn{2}{m{9cm}|}{20/02/2023}  \\
        \hline
         \textbf{Descripción} & \multicolumn{2}{m{9cm}|}{\textit{Como} estudiante de la URJC, \textit{quiero} iniciar sesión en el aula virtual,  \textit{para} acceder a mis cursos y recursos académicos.}  \\
        \hline
        \textbf{Criterios de aceptación} & \multicolumn{2}{m{9cm}|}{\begin{itemize}
        \item El usuario puede iniciar sesión con su usuario y contraseña.
        \item El sistema valida la autenticidad del usuario.
        \item El sistema muestra un mensaje de error si la autenticación falla.
        \end{itemize}}   \\
        \hline
    \end{tabular}
\end{table}

\end{document}
